%TEMPLATE-BEGIN 1
\documentclass[uplatex,dvipdfmx]{jsarticle} \usepackage{amsmath,amssymb,bm}
\usepackage{verbatim} \usepackage{svg} \usepackage{graphicx}
\usepackage[dvipdfmx]{hyperref} \usepackage{pxjahyper}
\usepackage{paracol} \usepackage{threeparttable}
\usepackage{seqsplit}
%TEMPLATE-END
\title{Deep Neural Network - Backpropagation with ReLU} \author{} \date{}
\begin{document}
\maketitle

\section*{user1157751}
\begin{paracol}{2}
I'm having some difficulty in deriving back propagation with ReLU, and I did some work, but I'm not sure if I'm on the right track.
\switchcolumn
ReLUについての誤差逆伝搬法を微分するのに私は少し手間取っている。多少やってみたのだが、正しい方法でできているのか不安だ。
\end{paracol}
\vspace{\baselineskip}
\begin{paracol}{2}
Cost Function: $\frac12(y-\hat{y})^2$ where $y$ is the real value, and $\hat{y}$ is a predicted value.

Also assume that $x>0$ always.
\switchcolumn
$y$を実数とし、$\hat{y}$を予測された値として、$\frac12(y-\hat{y})^2$がコスト関数だ。

また、常に$x>0$だと前提している。
\end{paracol}
\vspace{\baselineskip}
\begin{paracol}{2}
1 Layer ReLU, where the weight at the 1st layer is $w_1$

$$\frac{dC}{dw_1}=\frac{dC}{dR}\frac{dR}{dw_1}$$
$$\frac{dC}{w_1}=(y-ReLU(w_1x))(x)$$
\switchcolumn
1層のReLU。その第1層の重みは$w_1$である。
\end{paracol}



\section*{Neil Slater}
\begin{paracol}{2}
Working definitions of ReLU function and its derivative:

$$ReLU(x) = \begin{cases}
0,& \text{if } x<0\\
x,& \text{otherwise}
\end{cases}$$

$$\frac{d}{dx}ReLU(x) = \begin{cases}
0,& \text{if } x<0\\
1,& \text{otherwise}
\end{cases}$$
\switchcolumn
ReLu関数とその導関数の動作する定義は次のものだ:
\end{paracol}
\vspace{\baselineskip}
\begin{paracol}{2}
The derivative is the unit step function.

This does ignore a problem at $x=0$, where the gradient is not strictly defined, but that is not a practical concern for neural networks.

With the above formula, the derivative at 0 is 1, but you could equally treat it as 0, or 0.5 with no real impact to neural network performance.
\switchcolumn
この導関数は、単位階段関数だ。

これは、$x=0$で勾配が厳密に定義されていないという問題を無視する。しかし、ニューラルネットワークにおいてはそれは実際上は問題ではない。

上の公式によると、0における微分係数は1だ。しかし0や0.5としても、ニューラルネットワークのパフォーマンスにはどうという影響はなく同じだ。
\end{paracol}
\section*{Simplified network 単純化されたネットワーク}
\begin{paracol}{2}
With those definitions, let's take a look at your example networks.
\switchcolumn
これらの定義に則り、あなたが例示したネットワークらを見ていこう。
\end{paracol}
\vspace{\baselineskip}
\begin{paracol}{2}
You are running regression with cost function $C=\frac12(y-\hat{y})^2$.

You have defined $R$ as the output of the artificial neuron, but you have not defined an input value.

I'll add that for completeness - call it $z$, add some indexing by layer, and I prefer lower-case for the vectors and uppercase for matrices, so $r^{(1)}$ output of the first layer, $z^{(1)}$ for its input and $W^{(0)}$ for the weight connecting the neuron to its input $x$ (in a larger network, that might connect to a deeper $r$ value instead).

I have also adjusted the index number for the weight matrix -- why that is will become clearer for the larger network.

NB I am ignoring having more than newuron in each layer for now.
\switchcolumn
あなたは、$C=\frac12(y-\hat{y})^2$を費用関数として回帰を行っている。

あなたは$R$を、人工的ニューロンの出力として定義した。しかしまだ入力値を定義していない。

完全にするために私はそれを追加し、$z$と呼ぼう。層によって添字をつけよう。また私は、ベクトルについて小文字を、行列について大文字を使いたいから、 第1層の出力を$r^{(1)}$ と呼び、それへの入力を$z^{(1)}$と呼び、
\end{paracol}




\vspace{\baselineskip}
\begin{paracol}{2}
\switchcolumn
\end{paracol}
\end{document}
