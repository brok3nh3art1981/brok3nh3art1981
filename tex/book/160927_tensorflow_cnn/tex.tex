%TEMPLATE-BEGIN 1
\documentclass[uplatex,dvipdfmx]{jsarticle} \usepackage{amsmath,amssymb,bm}
\usepackage{verbatim} \usepackage{svg} \usepackage{graphicx}
\usepackage[dvipdfmx]{hyperref} \usepackage{pxjahyper}
\usepackage{paracol} \usepackage{threeparttable}
\usepackage{seqsplit}
%TEMPLATE-END
\title{『TensorFlowで学ぶディープラーニング入門\\畳み込みニューラルネットワーク徹底解説』\\読書ノート} \author{} \date{}
\begin{document}
\maketitle
中井悦司(なかい えつじ)による2016年9月27日の本。

\verb|https://github.com/enakai00/jupyter_tfbook|



\section*{p. 19}
4次関数によるモデル:
$$ y=w_0+w_1x+w_2x^2+w_3x^3+w_4x^4$$

ここで$x$は月であり$x\in\{1, 2, 3, 4, 5, 6, 7, 8, 9, 10, 11, 12\}$。$y$は$x$月について予測される月中平均気温(℃)。$w_0, w_1, w_2, w_3, w_4$はパラメータ。



\section*{p. 20}
誤差関数(error function)として二乗誤差(squared error)を考える:
$$E=\frac12\sum_{n=1}^{12} (y_n-t_n)^2$$

ここで$y_n$は$n$月の予測気温。$t_n$は$n$月の実際の気温である。



\section*{p. 22}
直線を表す数式:
$$f(x_1,x_2)=w_0+w_1x_1+w_2x_2=0$$

この形だと有名な$y=ax+b$よりも$x_1$と$x_2$を対称に扱える。

$f(x_1,x_2)=0$が境界となるほか、境界から離れるほど$f(x_1,x_2)$の値が$\pm\infty$に近づく。



\section*{p. 23}
検査結果$(x_1,x_2)$のときの感染確率$P(x_1,x_2)$をsigmoid関数$\sigma$を使って表す:
$$P(x_1,x_2)=\sigma(f(x_1,x_2))$$



\section*{p. 24}
1層のノードからなるニューラルネットワークの例。
$$z=\sigma(f(x_1,x_2))$$
where
$$f(x_1,x_2)=w_0+w_1x_1+w_2x_2$$



\section*{p. 25}
2層のノードからなるニューラルネットワークの例。
$$z=\sigma(f(z_1,z_2))$$
where
\begin{align*}
z_1 &= \sigma(f_1(x_1,x_2)) \\
z_2 &= \sigma(f_2(x_1,x_2))
\end{align*}
where
\begin{align*}
f_1(x_1,x_2) &= w_{10}+w_{11}x_1+w_{12}x_2 \\
f_2(x_1,x_2) &= w_{20}+w_{21}x_1+w_{22}x_2 \\
f(z_1,z_2) &= w_0+w_1z_1+w_2z_2
\end{align*}

このニューラルネットワークには$w_n$という形の合計9個のパラメータが含まれている。

ニューラルネットワークをより複雑にするためにノードを増やす方法には次の2つがある。
\begin{itemize}
\item 層の個数を増加する
\item 層にあるノードの個数を増加する
\end{itemize}



\section*{p. 30}
$x$月と予測気温$y$との関係:
$$y=w_0+w_1x+w_2x^2+w_3x^3+w_4x^4$$

2乗誤差による誤差関数$E$。ただし$y_n$は$n$月の予測気温。$t_n$は$n$月の実際の気温。
$$E=\frac12\sum_{n=1}^{12}(y_n-t_n)^2$$

これらを合わせると$n$月の予測気温$y_n$を次のように書ける。
\begin{align*}
\bm{y} &= (y_1, y_2, \cdots, y_{11}, y_{12}) \\
y_n &= w_0+w_1n+w_2n^2+w_3n^3+w_4n^4 \\
&= w_0n^0+w_1n^1+w_2n^2+w_3n^3+w_4n^4 \\
&= \sum_{m=0}^4w_mn^m \\
y_n &= y_n(w_0,w_1,w_2,w_3,w_4) \\
\bm{y} &= \bm{y}(w_0,w_1,w_2,w_3,w_4)
\end{align*}

この$y_n$を上記の誤差関数$E$に代入して次の式を得る。
\begin{align*}
E(\bm{y}) &= \frac12\sum_{n=1}^{12}(y_n-t_n)^2 \\
E(w_0,w_1,w_2,w_3,w_4) &= \frac12\sum_{n=1}^{12}
\left(
\sum_{m=0}^4w_mn^m-t_n
\right)^2
\end{align*}

これを最小にする$w_0, \cdots, w_4$を求めたい。$w_0, \cdots, w_4$のそれぞれによる偏微分の値を0とした次の連立方程式を解くことになる。
$$\frac{\partial E}{\partial w_m}(w_0,w_1,w_2,w_3,w_4) = 0 \quad\text{for}\ m=0,\cdots,4$$



\section*{p. 31}
偏微分とは、複数の変数を持つ関数をある1つの変数で微分することである。1変数の関数$y=f(x)$について最小値や最大値を求めるときに、微分係数を0とした次の方程式を解くのと本質的には同じだ。
$$\frac{df}{dx}(x)=0$$

この式は、入力$x$において関数$f$がどう動くかを問題にしている。入力$x$の付近でわずかに$x$が動くときに関数$f$の微小な動きとの比率が0であるとは、$x$の変化が$f$に影響していないということだ。つまりグラフはそこにおいて極小値や極大値や停留値を取る。$x$をそのように制約して求めるものが与式である。



\section*{p. 32}
次の2次関数を考える。
$$h(x_1,x_2)=\frac14(x_1^2+x_2^2)$$

その偏微分は次のようになる。
\begin{align*}
\frac{\partial h}{\partial x_1}(x_1,x_2) &= \frac12x_1 \\
\frac{\partial h}{\partial x_2}(x_1,x_2) &= \frac12x_2
\end{align*}

このとき次のように書いて勾配ベクトルと呼ぶ。
$$\nabla h(x_1,x_2) =
\begin{pmatrix}
\frac{\partial h}{\partial x_1}(x_1,x_2) \\[2mm]
\frac{\partial h}{\partial x_2}(x_1,x_2)
\end{pmatrix}
=
\begin{pmatrix}
\frac12x_1 \\[2mm]
\frac12x_2
\end{pmatrix}
=\frac12
\begin{pmatrix}
x_1\\x_2
\end{pmatrix}
$$

勾配ベクトル$\nabla h(x_1, x_2)$の向きは、壁を登る方向に一致する。勾配ベクトルの大きさ$||\nabla h(x_1,x_2)||$は、壁を登る傾きに一致する。



\section*{p. 33}
勾配降下法:
$$\bm{x}_{\rm new} = \bm{x}-\nabla h$$



\section*{p. 34}
次の2つの2次関数、$h_1$と$h_2$を考える。
\begin{align*}
h_1(x_1,x_2) &= \frac34(x_1^2+x_2^2) \\
h_2(x_1,x_2) &= \frac54(x_1^2+x_2^2)
\end{align*}

それぞれの勾配ベクトルとして次のものが導ける。
\begin{align*}
\nabla h_1 = \frac32
\begin{pmatrix}
x_1\\x_2
\end{pmatrix} \\
\nabla h_2 = \frac52
\begin{pmatrix}
x_1\\x_2
\end{pmatrix}
\end{align*}

この後者について
$\bm{x}_{\rm new} = \bm{x}-\nabla h$
という素朴な勾配降下法で$\bm{x}$を更新していくと、パラメータ$\bm{x} = (x_1,x_2)$は発散する。

よって学習率$c$を用いて次のようにする。
$$\bm{x}_{\rm new} = \bm{x}-c\nabla h$$


二乗誤差$E(w_0,w_1,w_2,w_3,w_4) $を最小化するのにも同じことをする。
$$\bm{w}_{\rm new} = \bm{w}-\epsilon\nabla E(\bm{w})$$

ここで勾配ベクトル$\nabla E(\bm{w})$は次のものである。
$$\nabla E(\bm{w}) =
\begin{pmatrix}
\frac{\partial E}{\partial w_0}(\bm{w}) \\
\vdots \\
\frac{\partial E}{\partial w_4}(\bm{w})
\end{pmatrix}$$


\section*{p. 38 環境準備}
TensorFlowをインストール済みのDockerイメージが本書から提供されているので、それを使えばたいていの環境で同じコードを簡単に動かせるとのこと。

Ubuntu OSについて考える。

\begin{verbatim}
$ sudo apt install docker.io
$ docker --version
Docker version 18.09.2, build 6247962
\end{verbatim}

\begin{quotation}
In the past this way was discouraged as the docker package was super outdated. The universe sources are fairly recent now.
\end{quotation}

\begin{quotation}
This is now the best way nowadays, since the Ubuntu repos are keeping up with the docker releases.
\end{quotation}

\begin{verbatim}
# Docker HubでDockerイメージらを検索する。
sudo docker search enakai00
# DockerイメージをDocker Hubから取得する。
sudo docker pull enakai00/jupyter_tensorflow
# 取得済みのイメージらを一覧する。
sudo docker images
# Dockerコンテナを一覧する。
sudo docker ps     # 実行中のもののみ表示する
sudo docker ps -a  # 停止中のものも含めて表示する
# Dockerコンテナをシャットダウンする。
sudo docker stop example
# Dockerコンテナを削除する。
sudo docker rm example
\end{verbatim}

\begin{verbatim}
# コンテナを生成して起動する。
sudo docker run enakai00/jupyter_tensorflow
# 名前つきでコンテナを生成する。
sudo docker run --name nakai enakai00/jupyter_tensorflow
# ホスト側のポートをコンテナ側のポートに転送する。http://localhost:8888/ で繋がるようになる。
sudo docker run --name nakai -p 8888:8888 enakai00/jupyter_tensorflow
# ログイン用のパスワードを書籍の通りに指定する。
sudo docker run --name nakai -p 8888:8888 -e PASSWORD=passw0rd enakai00/jupyter_tensorflow
\end{verbatim}



\section*{p. 48 TensorFlowクイックツアー}
\begin{verbatim}
pip3 install tensorflow
\end{verbatim}



\section*{p. 59}
予測気温の式。
$$y(x)=w_0+w_1x+w_2x^2+w_3x^3+w_4x^4=\sum_{m=0}^4w_mx^m$$

\section*{p. 67}
$(x_1,x_2)$平面を分割する直線の式:
$$f(x_1,x_2)=w_0+w_1x_1+w_2x_2=0$$

この式は境界線から離れるにしたがって$\pm\infty$に向かって変化する。また、未知のパラメータ$w_0,w_1,w_2$を含んでいる。

sigmoid関数:
$$\sigma(x)=\frac1{1+e^{-x}}$$

検査結果$(x_1,x_2)$に対して、ウィルスに感染している確率:
$$P(x_1,x_2)=\sigma(f(x_1,x_2))$$

教師データを考える。$n$番目の入力を$(x_{1n},x_{2n})$とする。$n$番目の正解を$t_n=0,1$とする。感染している場合が$t_n=1$である。現在のモデルによると、感染している確率は$P(x_{1n},x_{2n})$となる。この値に応じて確率的に予測してみる。

$n$番目のデータを正しく予測する確率を$P_n$として、次が成り立つ。
\begin{itemize}
\item $t_n=1$の場合:$P_n=P(x_{1n},x_{2n})$
\item $t_n=0$の場合:$P_n=1-P(x_{1n},x_{2n})$
\end{itemize}

1つの数式にまとめる。
$$P_n = \{P(x_{1n},x_{2n})\}^{t_n} \{1-P(x_{1n},x_{2n})\}^{1-t_n}$$

$N$個のデータすべてに正解する確率$P$を考えたい。
$$P = P_1\times P_2\times\cdots\times P_N = \prod_{n=1}^NP_n$$
$$P=\prod_{n=1}^N \{P(x_{1n},x_{2n})\}^{t_n} \{1-P(x_{1n},x_{2n})\}^{1-t_n} $$

「与えられたデータを正しく予測する確率を最大化する」という方針でパラメータを調整する方法を「最尤推定法 (maximum likelihood estimation)」という。

掛け算を大量に含む数式はTensorFlowでは効率がよくない。よって次の誤差関数を使う:
$$E=-\log P$$

なお対数については次の操作ができる:
\begin{align*}
\log ab &= \log a+\log b \\
\log a^n &= n\log a
\end{align*}

これにより次の変形ができる。
\begin{align*}
E &= -\log P \\
&= -\log\prod_{n=1}^N \{P(x_{1n},x_{2n})\}^{t_n} \{1-P(x_{1n},x_{2n})\}^{1-t_n} \\
&= -\sum_{n=1}^N \log \left[ \{P(x_{1n},x_{2n})\}^{t_n} \{1-P(x_{1n},x_{2n})\}^{1-t_n} \right] \\
&= -\sum_{n=1}^N \left[ \log \{P(x_{1n},x_{2n})\}^{t_n} + \log \{1-P(x_{1n},x_{2n})\}^{1-t_n} \right] \\
&= -\sum_{n=1}^N \left[ t_n \log P(x_{1n},x_{2n}) + (1-t_n)\log \{1-P(x_{1n},x_{2n})\} \right] \\
\end{align*}






\section*{implementation}
\verbatiminput{input/p051.py}



\vspace{\baselineskip}
\begin{paracol}{2}
\switchcolumn
\end{paracol}
\end{document}










