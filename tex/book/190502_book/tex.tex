%TEMPLATE-BEGIN 1
\documentclass[uplatex,dvipdfmx]{jsarticle} \usepackage{amsmath,amssymb,bm}
\usepackage{verbatim} \usepackage{svg} \usepackage{graphicx}
\usepackage[dvipdfmx]{hyperref} \usepackage{pxjahyper}
\usepackage{paracol} \usepackage{threeparttable}
\usepackage{seqsplit}
%TEMPLATE-END
\title{読書ノート} \author{} \date{}
\begin{document}
\maketitle



\section*{2008年9月6日 細野真宏 細野真宏の数学嫌いでも「数学的思考力」が飛躍的に身に付く本!}

1969年生まれ、慶應大学理工学部、大学院卒。カードキャプターさくらのケロちゃんのようなイラストが多数掲載されていてかなりかわいい。人間の思考のあり方を論じたような、かなり本質的な議題の本。意味のある内容だと思うが、個人的には得たものはなかった。
\vspace{\baselineskip}
\begin{paracol}{2}
\switchcolumn
\end{paracol}
\end{document}










