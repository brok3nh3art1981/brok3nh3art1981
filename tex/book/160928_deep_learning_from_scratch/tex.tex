%TEMPLATE-BEGIN 1
\documentclass[uplatex,dvipdfmx]{jsarticle} \usepackage{amsmath,amssymb,bm}
\usepackage{verbatim} \usepackage{svg} \usepackage{graphicx}
\usepackage[dvipdfmx]{hyperref} \usepackage{pxjahyper}
\usepackage{paracol} \usepackage{threeparttable}
\usepackage{seqsplit}
%TEMPLATE-END
\title{『ゼロから作るDeep Learning\\Pythonで学ぶディープラーニングの理論と実装』\\読書ノート} \author{} \date{}
\begin{document}
\maketitle
斎藤康毅(さいとう こうき。1984-)による2016年9月28日の本である。

ソースコード: https://github.com/oreilly-japan/deep-learning-from-scratch

deep-learning-from-scratch/.gitignoreに.DS\_Storeとある。これは何? macOSのフォルダの状態データ。

Pythonの文法チェッカーとしては{\tt flake8}コマンドがある。
\section*{p. 2}
ディープラーニングのフレームワークの例: Caffe、TensorFlow、Chainer、Theano。
\section*{p. 20}
『入門Python 3』、『Pythonによるデータ分析入門』という本がある。『Scipy Lecture Notes』というウェブサイトがある。.pdfファイルで全690ページ。
\section*{p. 21}
Frank Rosenblatt (1928-1971)。論文『The Perceptron: A Probabilistic Model for Information Storage and Organization in the Brain』(1958)がある。全23ページ。
\begin{quotation}
Prefiguring Cyberculture: An Intellectual History (2004-09-17):

Shortly afterwards (July 11, 1971) Frank Rosenblatt, its guiding inspiration, died in a boating accident that some say was a suicide.

But within the AI community, early connectionism had already met its demise (see, for example, Dreyfus and Dreyfus 1988, 24).

In the absense of funding, and of the kinds of concrete successes in embodying biological principles in non-biological systems that might have generated new alliances and new sources of funding, the union promised between organisms and computers failed to materialize.

Neural net models, which had been the heart of Rosenblatt's effort, had to wait another 15 years to be revived; interactionist models of development have had to wait even longer.

Evidently, the soil of American science in the 1960s was not quite right for this largely European transplant to take root.

Even so, as Hubert and Stuart Dreyfus write (1988, 24), ``blaming the rout of the connectionists on an antiholistic bias is too simple.''

In a rather expansive sweep, they fault the influence of an atomistic tradition of Western philosophy, from Plato to Kant; von Foerster (1995), by contrast, blames both the funding policies of American science (the short fuding cycle, the emphasis on targeted research) and the excessively narrow training of American students.

(Historians have yet to have their say on the matter.)
\end{quotation}
\section*{p. 22}
$$y = 
\begin{cases}
0 & \text{if\quad} w_1x_1+w_2x_2\leq\theta\\
1 & \text{if\quad} w_1x_1+w_2x_2>\theta
\end{cases}
$$

\section*{p. 23}
ANDゲートが、例えば$(w_1,w_2,\theta)=(0.5,0.5,0.7)$で実現できると記載がある。matplotlibでプロットしてみる。

\verbatiminput{input/001/p023_plot_and.py}

\begin{figure} \centering
\includegraphics[width=100mm]{input/p023.pdf}
\caption{$(w_1,w_2,\theta)=(0.5,0.5,0.7)$によるANDゲート}
\end{figure}

\section*{p. 26}
$$y = 
\begin{cases}
0 & \text{if\quad} b+w_1x_1+w_2x_2\leq0\\
1 & \text{if\quad} b+w_1x_1+w_2x_2>0
\end{cases}
$$


\begin{figure} \centering
\begin{tabular}{c}
\begin{minipage}{0.3\hsize}
\includegraphics[width=50mm]{input/p026_1.pdf}
\end{minipage}
\begin{minipage}{0.3\hsize}
\includegraphics[width=50mm]{input/p026_2.pdf}
\end{minipage}
\begin{minipage}{0.3\hsize}
\includegraphics[width=50mm]{input/p026_3.pdf}
\end{minipage}


\end{tabular}
\caption{バイアスを用いたAND、NAND、ORゲート}
\end{figure}

\section*{p. 42}
$$y=h(b+w_1x_1+w_2x_2)$$
$$h(x) = 
\begin{cases}
0 & \text{if\quad} x\leq0\\
1 & \text{if\quad} x>0
\end{cases}
$$

これはbinary step関数とか、ヘヴィサイドの階段関数と呼ばれるようである。
\section*{p. 45}
sigmoid function。シグモイドとは、ギリシャ語の文字シグマのようにうねってるということ。
$$ h(x) = \frac{1}{1+e^{-x}} $$

\section*{p. 67}
softmax function.

$$ y_k = \frac{e^{a_k}}{\sum\limits_{i=1}^n e^{a_i}} $$


\section*{p. 73}
MNIST Databaseのデータをダウンロードなどしてくれるmnist.pyを本書が提供している。実行時の表示を以下にメモしておく。

\begin{verbatim}
$ python3 mnist.py 
Downloading train-images-idx3-ubyte.gz ... 
Done
Downloading train-labels-idx1-ubyte.gz ... 
Done
Downloading t10k-images-idx3-ubyte.gz ... 
Done
Downloading t10k-labels-idx1-ubyte.gz ... 
Done
Converting train-images-idx3-ubyte.gz to NumPy Array ...
Done
Converting train-labels-idx1-ubyte.gz to NumPy Array ...
Done
Converting t10k-images-idx3-ubyte.gz to NumPy Array ...
Done
Converting t10k-labels-idx1-ubyte.gz to NumPy Array ...
Done
Creating pickle file ...
Done!
\end{verbatim}


\section*{p. 75}
学習済みパラメータを使ってMNIST Databaseのテスト用データを判定させる。正答率0.9352が得られる。第0層から第3層までを用いる3層のニューラルネットワークを用いる。各層のニューロンの個数は、$28\times 28=784$、50、100、$10=|\{0, 1, 2, 3, 4, 5, 6, 7, 8, 9\}|$であり、合計で$784+50+100+10=944$個だ。ここで隠れ層のニューロンの個数50と100は自由らしいが、学習済みパラメータを使うので動かすわけにいかない。なお、学習済みデータはそれなりに大きい。どのくらいの大きさだろうか。

第1層の第0ニューロンを考えてみる。すると、バイアス1つと、784個の入力が入ってくる。よって重みの個数は$784+1=785$だろう。第1層の50個のニューロンについてこれは同様に言えるから、合計で、$50\times 785=39250$個の重みがあると考えられる。次に第2層の第0ニューロンを考えてみる。すると入力の個数は51個だろう。第2層の100個のニューロンについてこれは同様に言えるから、合計で、$100\times 51=5100$個の重みがあると考えられる。第3層の第0ニューロンについて考えてみる。すると入力の個数は101だろう。第3層の10個のニューロンについてこれは同様に言えるから、合計で、$10\times 101=1010$個の重みがあると考えられる。こうして得られた値の総和を求めると、$39250+5100+1010=45360$個の重みがあるということになる。

p. 76のコードに次の美しくないコードを挿入してこの個数(45360)を確認した。{\tt print(sum(map(len, [W1.flatten(), W2.flatten(), W3.flatten(), b1.flatten(), b2.flatten(), b3.flatten()])))} また、同じコードに{\tt f.write(str(network))}などと挿入して書き出したテキストファイルのサイズは895,218バイトであったが、これを重みの個数45360で割ると、19.73584656084656が得られる。1つの重み当たりテキストで20文字弱を占めているということになり、大きな矛盾はない。

ニューロンが944個でシナプスが45360個だと考えられるだろう。$\frac{944}{45360}=0.020811...$であり、$\frac{45360}{944}=48.050847...$である。つまり1個のニューロンにつき50個弱のシナプスが存在する。これはもちろん、ネットワークの構成によるのだろう。なおおそらく、MNISTは最小の課題の一つだろう。ニューロンとシナプスが約1000個と約50000個だと思ってしまおうか。

テスト用データの個数は10000個で、自分の環境でそれを処理するのは2秒ほどかかった。つまり、0.2 msで1個処理でき、1 msで5個処理できる。学習との計算量の比率はかなり異なると思っていいのだろう。多くの用途において、学習済みデータを用いた推論は、スマートフォンや組み込み機器のハードウェアで処理できるという印象は感じられた。

なお、本書で提供されている学習済みパラメータの.pklファイルであるch03/sample\_weight.pklは181,853バイトだ。素朴にテキスト化したところ上記の通り895,218バイトだったわけである。$\frac{895,218}{181,853}=4.922756...$で5倍弱になっている。しかも精度が落ちているかもしれない。深層学習などを扱う上では、このようなデータを格納しておく必要性はしばしば生じそうだ。.pklは実用的にもよい選択肢なのだろうか。.zipにしてみるとそれぞれ、169,644バイトと297,814バイトであった。もし.pkl表現に後方互換性がなかったら、テキストファイルを.zipにすることは大きな損ではないかもしれない。.json形式が互換性が高いだろう。.csv形式が人気があるようだ。思うに、.csv形式は行で貪欲に完結しているのは長所だろう。

\section*{p. 88}
人気のある損失関数である2乗和誤差(mean squared error)。2で割るのは、微分したときに係数が消える利点のため。
$$E=\frac12\sum_k (y_k-t_k)^2$$

また別の人気のある損失関数である交差エントロピー誤差(cross entropy error)。
$$E=-\sum_k t_k \log y_k$$

N個の訓練データの損失関数の和。
\begin{align*}
E &=\frac1N\sum_n \left( -\sum_k t_{nk} \log y_{nk} \right) \\
&=-\frac1N\sum_n\sum_k t_{nk}\log y_{nk}
\end{align*}

\section*{p. 97}
微分。前方差分による。
$$\frac{df(x)}{dx}=\lim_{h\to0}\frac{f(x+h)-f(x)}{h}$$


\section*{p. 101}
サンプルプログラムch04/gradient\_1d.pyのコードスニペットについて考える。
\begin{verbatim}
def tangent_line(f, x):
    """関数fの入力xにおける接線を返す。"""
    d = numerical_diff(f, x)  # xにおける微分係数を得る
    print(d)                  # 0.1999999999990898
    y = f(x) - d*x            # 高さはxのときf(x)だが0のときはdx減る
    return lambda t: d*t + y  # y = ax + bならぬdt + yを返す
\end{verbatim}


\section*{p. 107}
勾配法。
$$x_0=x_0-\eta\frac{\partial f}{\partial x_0}$$
$$x_1=x_1-\eta\frac{\partial f}{\partial x_1}$$

\section*{p. 108}
学習率が大きいからといって発散するものだろうか。またそもそも、勾配は方向を知るのにのみ使って、その大きさは用いなくてもよいのではないか。むしろ平坦な場所ほど大きく移動したい気もする。

\section*{p. 118}
のコードで学習が実行される。ここまでの本文では誤差逆伝搬法の内容の説明はないが、効果は同じということで、コードでは誤差逆伝搬法による方法がデフォルトでオンになっている。それでも少し時間がかかる。誤差逆伝搬法を使わない場合には100倍くらいかかるだろうか。

どこでどの程度の時間がかかっているのだろう。巨大な行列の乗算が繰り返されるが、そもそも、行列の積を素朴に求めるときの計算量は何だろうか。$L\times M$行列$A$と$M\times N$行列$B$の積$C$を考えてみようか。Cに存在するLN個の要素それぞれは、AとBのそれぞれM個の要素について、各々を掛け、足し合わせたものである。よって、$LN(M+(M-1))$だと考えられる。大雑把に見れば、$LMN$だ。つまり、行列A、Bの辺の長さがどれも計算量に同様に作用している。

第0層入力層に$28\times28=784$個のニューロンがあり、第1層隠れ層に50個のニューロンがあるとする。すると第0層と第1層を結ぶ辺は第1層のニューロン1個につき784個あるから合計で$50\times 784=39,200$個だろう。行列の積として考えると、$1\times784$行列$A$と$784\times 50$行列$B$の積$C$が$1\times 50$行列なのだと言える。$C$の各要素は784要素の積和なので全体の計算量が$1\times 784\times 50$という感じだろう。そして、実際に入力されるのはミニバッチである。

複数の行列の積$ABCDEFG$のようなものを考えてみる。結合法則により乗算の順序は任意である。小さいものからやったほうが計算量が小さい気がする。しかし、ニューラルネットワークはsigmoid関数などの活性化関数で分断されているので、関係はないと思われる。

活性化関数では累乗の計算がある。累乗の計算量は$O(\log n)$だったろう。とりあえず考えずにおこう。ミニバッチの大きさを100と考える。行列として書くと、$100\times78
どこでどの程度の時間がかかっているのだろう。巨大な行列の乗算4$行列$A$と$784\times50$行列$B$と$50\times10$行列$C$から$100\times10$行列$D$を得ると考えられる。まず第1層の計算のために、$100\times784\times50=3,920,000$回四則演算して$100\times50$行列を得る。次に第2層を算出するために、$100\times50\times10=50,000$回四則演算して$100\times10$行列を得る。よって1個のミニバッチをpredict()するのに$3,920,000+50,000=3,970,000$回四則演算すると思われる。

MNISTの訓練データは60,000個あるから、ミニバッチの大きさを100としたときには、$\frac{60,000}{100}=600$回が1 epochをなすのだろう。1 epochあたりの四則演算の回数は、$3,970,000\times600=2,382,000,000$だと概算される。勾配などの計算を考えずにである。すでに$10^9$のオーダーであるから、1分間かかるくらいの感じだろうか。本書のサンプルtrain\_neuralnet.pyにおいては、epoch=600で大きさ100のミニバッチを10,000回実行しているから、epoch=$\frac{10,000}{600}=16.\bar{6}$である。さほど時間はかからず手元の環境で45秒間ほどである。なお、numpyモジュールは複数のCPUコアを自動的に活用しているようである。

数値差分により勾配を得る方法、それは重みつまり辺のすべてについて上下に動かし、そのたびにpredict()するのだから、大きな計算量になるとは想像できる。

例えば本書サンプルtrain\_neuralnet.pyで言えば、iters\_num=10,000回ミニバッチを処理するたびにTwoLayerNet\#numerical\_gradient()しているが、TwoLayerNet\#numerical\_gradient()はそれぞれの重みのセットについてnumerical\_gradient()を呼び、numerical\_gradient()は$2\times3=6$回TwoLayerNet\#loss()を呼び、loss()はpredict()を呼んでいる。

iter\_num=1としてpredict()が実行された回数を数えてみると79,523である。辺の数が$39,200+500=39,700$であったから、$2\times39,700=79,400$がほぼ近い。バイアスも数え入れて数え直そうか。$50\times785=39,250$、$10\times51=510$、$39,250+510=39,760$、$2\times39,760=79,520$となる。残りの3つについては、TwoLayerNetの外からloss()が1回、accuracy()が2回呼ばれているので、数が合うと考えられる。

\section*{p. 141}
ReLU (rectified linear unit):
$$y = 
\begin{cases}
x & \text{if\quad} x>0\\
0 & \text{if\quad} x\leq0
\end{cases}
$$
その微分:
$$\frac{\partial y}{\partial x} = 
\begin{cases}
1 & \text{if\quad} x>0\\
0 & \text{if\quad} x\leq0
\end{cases}
$$

\section*{p. 144}
シグモイド関数の微分。
$$sigmoid(x)=\frac{1}{1+e^{-x}}$$

これを、$xy$と$e^x$と$x+y$と$\frac{x}y$に分けて考える。
$$\frac{\partial L}{\partial y}(-y^2)\times1\times e^{-x}\times(-1) = \frac{\partial L}{\partial y}y^2 e^{-x}$$

シグモイド関数の微分が得られた。これをさらに整理する:
\begin{align*}
\frac{\partial L}{\partial y}y^2 e^{-x} &= \frac{\partial L}{\partial y} \left( \frac{1}{1+e^{-x}} \right)^2 e^{-x} \\
&=  \frac{\partial L}{\partial y} \frac{1}{1+e^{-x}} \frac{e^{-x}}{1+e^{-x}} \\
&=  \frac{\partial L}{\partial y} \frac{e^{-x}}{(1+e^{-x})^2} \\
&=  \frac{\partial L}{\partial y} \left( \frac{1+e^{-x}}{(1+e^{-x})^2} - \frac{1}{(1+e^{-x})^2} \right) \\
&=  \frac{\partial L}{\partial y} \left( \frac{1}{1+e^{-x}} - \frac{1}{(1+e^{-x})^2} \right) \\
&=  \frac{\partial L}{\partial y} \frac{1}{1+e^{-x}} \left( 1 - \frac{1}{1+e^{-x}} \right) \\
&=  \frac{\partial L}{\partial y} y (1-y) \\
\end{align*}


\vspace{\baselineskip}
\begin{paracol}{2}
\switchcolumn
\end{paracol}
\end{document}










