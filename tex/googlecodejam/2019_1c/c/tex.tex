%TEMPLATE-BEGIN 1
\documentclass[uplatex,dvipdfmx]{jsarticle} \usepackage{amsmath,amssymb,bm}
\usepackage{verbatim} \usepackage{svg} \usepackage{graphicx}
\usepackage[dvipdfmx]{hyperref} \usepackage{pxjahyper}
\usepackage{paracol} \usepackage{threeparttable}
\usepackage{seqsplit}
%TEMPLATE-END
\title{Bacterial Tactics 細菌の戦術} \author{} \date{}
\begin{document}
\maketitle
\section*{Problem 問題}
\begin{paracol}{2}
Becca and Terry are microbiologists who have a friendly rivalry.

When they need a break from their research, they like to play a game together.

The game is played on a matrix of unit cells with {\bf R} rows and {\bf C} columns.

Initially, each cell is either empty, or contains radioactive material.
\switchcolumn
ベッカとテリーは微生物学者であり、仲の良い競争相手だ。

研究の休憩のときには、一緒にゲームをするのが好きである。

ゲームは、セルを単位とする{\bf R}行{\bf C}列の行列において行われる。

最初、それぞれのセルは空か、あるいは放射線物質を含んでいる。
\end{paracol}
\vspace{\baselineskip}
\begin{paracol}{2}
On each player's turn, if there are no empty cells in the matrix, that player loses the game.

Otherwise, they choose an empty cell and place a colony of bacteria there.

Bacteria colonies come in two types: H (for ``horizontal'') and V(for ``vertical'').
\switchcolumn
各プレイヤーのターンにおいて、行列に空のセルがなければ、そのプレイヤーはゲームに負けである。

そうでないならば、プレイヤーは空のセルを選び、そこに細菌のコロニーを置く。

細菌のコロニーには次の2種類がある。(水平を意味する)Hと、(垂直を意味する)Vである。
\end{paracol}
\vspace{\baselineskip}
\begin{paracol}{2}
\begin{itemize}
\item When a type H colony is placed into an empty cell, it occupies that cell (making it non-empty), and also tries to spread into the cell immediately to the west (if there is one) and the cell immediately to the east (if there is one).
\item When a type V coloney is placed into an empty cell, it occupies that cell (making it non-empty), and also tries to spread into the cell immediately to the south (if there is one) and the cell immediately to the north (if there is one).
\end{itemize}
\switchcolumn
\begin{itemize}
\item もしもタイプHのコロニーが空のセルに置かれると、そのセルを占有する。(つまり、そのセルを空でなくする。) そして、(もし西にセルがあれば)すぐ西のセルに、また(もし東にセルがあれば)すぐ東のセルに広がろうとする。
\item もしもタイプVのコロニーが空のセルに置かれると、そのセルを占有する。(つまり、そのセルを空でなくする。) そして、(もし南にセルがあれば)すぐ南のセルに、また(もし北にセルがあれば)すぐ北のセルに広がろうとする。
\end{itemize}
\end{paracol}
\vspace{\baselineskip}
\begin{paracol}{2}
Whenever a colony (of either type) tries to spread into a cell:
\switchcolumn
(タイプによらず)コロニーがあるセルに広がるときには次のことが起こる:
\end{paracol}
\vspace{\baselineskip}
\begin{paracol}{2}
\begin{itemize}
\item If the cell contains radioactive material, the colony mutates and the player who placed the colony loses the game.
\item If that cell is empty, the colony occupies that cell (making it non-empty), and then the rule above is triggered again (i.e. the colony will try to spread further).
\item If the cell already contains bacteria (of any type), the colony does not spread into that cell.
\end{itemize}
\switchcolumn
\begin{itemize}
\item もしセルに放射線物質があれば、コロニーは変異し、そのコロニーを置いたプレイヤーはゲームに負ける。
\item もしセルが空ならば、コロニーはそのセルを支配する。(空でなくする。) そして、上の規則が再び起こる。(つまり、コロニーはさらに広がろうとする。)
\item もしセルにすでに(どちらかのタイプの)細菌があれば、コロニーはそのセルには広がらない。
\end{itemize}
\end{paracol}
\vspace{\baselineskip}
\begin{paracol}{2}
Notice that it may be possible that all of a player's available moves would cause them to lose the game, and so they are doomed.

See the sample case explanations below for examples of how the game works.
\switchcolumn
なお、あるプレイヤーに可能なすべての動きが、そのプレイヤーを敗北させる場合はある。そのときは破滅の運命だ。

下で説明されているサンプルケースには、ゲームがどのようなものか、例が示されている。
\end{paracol}
\vspace{\baselineskip}
\begin{paracol}{2}
Becca makes the first move, and then the two players alternate moves until one of them loses the game.

If both players play optimally, who will win?

And, if Becca will win, how many distinct winning opening moves does she have?

(Two opening moves are distinct if and only if either use different cells, or different kinds of colony, or both.)
\switchcolumn
ベッカが最初に動く。以降、どちらかのプレイヤーがゲームに負けるまで、2人のプレイヤーが順番に動く。

両者の動きがどちらも最善なら、どちらが勝つ?

そしてもしベッカが勝つなら、彼女の可能な異なる最初の動きはいくつあったか?

(2つの最初の動きは、異なるセルを使っているか、異なる種類のコロニーか、あるいはその両方のときにのみ、区別される。)
\end{paracol}
\subsection*{Input 入力}
\begin{paracol}{2}
The first line of the input gives the number of test cases, {\bf T}.

{\bf T} test cases follow.

Each case begins with one line containing two integers {\bf R} and {\bf C}: the number of rows and columns, respectively, in the matrix.

Then, there are {\bf R} more rows of {\bf C} characters each.

The j-th character on the i-th of these lines represents the j-th column of the i-th row of the matrix.

Each character is either . (an emtpy cell) or \# (a cell with radioactive material).
\switchcolumn
入力の最初の行は、テストケースの個数{\bf T}をもたらす。

{\bf T}個のテストケースらがつづく。

それぞれのケースは、2つの整数{\bf R}と{\bf C}を持つ1行で始まる。それらはそれぞれ、行列の行数と列数である。

そして、それぞれの行が{\bf C}文字ある、さらに{\bf R}行が与えられる。

それらの行のなかで、i番目の行にあるj個目の文字は、行列における、i番目の行のj番目の列を表している。

それぞれの文字は、(空のセルを表す).か、(放射線物質のセルを表す)\#である。
\end{paracol}
\subsection*{Output 出力}
\begin{paracol}{2}
For each test cases, output one line containing {\tt Case \#x:\;y}, where x is the test case number (starting from 1), and y is an integer: either 0 if Becca will not win, or, if Becca will win, the number of distinct winning opening moves she can make, as described above.
\switchcolumn
各テストケースについて、{\tt Case \#x:\;y}という1行を出力せよ。ここでxは、(1から始まる)テストケースの番号であり、yは次のような整数だ。Beccaが勝たないときはyは0であり、Beccaが勝つときは、上に述べたように、勝利できる異なる最初の動きの数がyである。
\end{paracol}
\subsection*{Limits 制約}
\begin{paracol}{2}
Time limit: 30 seconds per test set.

Memory limit: 1GB

$1 \leq \mathbf{T} \leq 100$.
\switchcolumn
時間制限: テストケース1つにつき30秒。

メモリ制限: 1 GB

$1 \leq \mathbf{T} \leq 100$。
\end{paracol}
\subsection*{Test set 1(Visible) テストセット1(可視)}
\begin{paracol}{2}
$1 \leq \mathbf{R} \leq 4$.

$1 \leq \mathbf{C} \leq 4$.
\switchcolumn
$1 \leq \mathbf{R} \leq 4$。

$1 \leq \mathbf{C} \leq 4$。
\end{paracol}
\subsection*{Test set 2 (Hidden) テストセット2(不可視)}
\begin{paracol}{2}
$1 \leq \mathbf{R} \leq 15$.

$1 \leq \mathbf{C} \leq 15$.
\switchcolumn
$1 \leq \mathbf{R} \leq 15$。

$1 \leq \mathbf{C} \leq 15$。
\end{paracol}
\subsection*{Sample サンプル}
\begin{paracol}{2}
Input

\begin{verbatim}
5
2 2
..
.#
4 4
.#..
..#.
#...
...#
3 4
#.##
....
#.##
1 1
.
1 2
##
\end{verbatim}
\switchcolumn
Output

\begin{verbatim}







Case #1: 0
Case #2: 0
Case #3: 7
Case #4: 2
Case #5: 0
\end{verbatim}
\end{paracol}
\vspace{\baselineskip}
\begin{paracol}{2}
In Sample Case \#1, Becca cannot place an H colony in the southwest empty cell or a V colony in the northeast empty cell, because those would lose.

She has only two possible strategies that do not cause her to lose immediately:
\switchcolumn
サンプルケース\#1では、ベッカは、南西の空のセルにはHコロニーを置くことはできず、北東のセルにVコロニーを置くことはできない。そうすれば負けてしまうからだ。

彼女がただちに負けないためには、次の2つの可能な戦略だけが存在する:
\end{paracol}
\vspace{\baselineskip}
\begin{paracol}{2}
\begin{enumerate}
\item Place an H colony in the northwest or northeast empty cells. The colony will also spread to the other of those two cells.
\item Place a V colony in the northwest or southwest empty cell. The colony will also spread to the other of those two cells.

\end{enumerate}
\switchcolumn
\begin{enumerate}
\item Hコロニーを北西か北東の空のセルに置く。そのコロニーには、それら2つのセルのもう片方にも広がることになる。
\item Vコロニーを北西か南西の空のセルに置く。そのコロニーには、それら2つのセルのもう片方にも広がることになる。
\end{enumerate}
\end{paracol}
\vspace{\baselineskip}
\begin{paracol}{2}
In Sample Case \#2, any of Becca's opening moves would cause a mutation.
\switchcolumn
サンプルケース\#2では、Beccaのどの最初の動きも、変異を起こす。
\end{paracol}
\vspace{\baselineskip}
\begin{paracol}{2}
In Sample Case \#3, five of Becca's possible opening moves would cause a mutation, but the other seven are winning.

She can place an H colony in any of the cells of the second row, or she can place a V colony in any of the cells of the second column.

In either case, she leaves two disconnected sets of 1 or 2 cells each.

In each of those sets, only one type of colony can be played, and playing it consumes all of the empty cells in that set.

So whichever of those sets Terry chooses to consume, Becca can consume the other, leaving Terry with no moves.
\switchcolumn
サンプルケース\#3では、ベッカの可能な最初の動きのうち5個は変異を引き起こす。だが他の7個は勝利をもたらすものだ。

彼女は2行目の任意のセルにHコロニーを置けるし、そうではなく、2列目の任意のセルにVコロニーを置くこともできる。

どちらの場合にも、1つまたは2つのセルからなる互いに接続しない2つの集まりができる。

そのどのセットについても、1種類のコロニーしか置くことができない。そして置けるコロニーを置くと、そのセットの空のセルは使い果たされる。

よって、テリーがそのどのセットを選ぶとしても、ベッカはもう片方を使い果たし、テリーの置き場をなくすことができる。
\end{paracol}
\vspace{\baselineskip}
\begin{paracol}{2}
In Sample Case \#4, both of Becca's two distinct possible opening moves are winning.
\switchcolumn
サンプルケース\#4では、存在するベッカの可能な最初の動きはどちらも勝利をもたらす。
\end{paracol}
\vspace{\baselineskip}
\begin{paracol}{2}
In Sample Case \#5, Becca has no possible opening moves.
\switchcolumn
サンプルケース\#5では、ベッカには可能な最初の動きがない。
\end{paracol}
\section*{Analysis 解析}
\subsection*{Test set 1 テストセット 1}
\begin{paracol}{2}
On a player's turn, the grid will be in some state, according to whether any bacteria have already been placed, and how they have spread.

We can determine whether a state is {\it losing} via the following recursive definition:

\begin{itemize}
\item the player has no moves because there  are no empty cells
\item any of the player's moves would either cause a mutation, or give the opponent a {\it winning} state.
\end{itemize}

Observe that if a state is not losing, it must be winning, since it has at least one move that does not cause a mutation and does not give the opponent a winning state.
\switchcolumn
プレイヤーのターンのときに、細菌がどのように配置され、そしてどのように広がったかによって、グリッドはある状態にある。

私達は、次の再帰的な定義を用いて、ある状態が{\bf 負けている}かを決定できる。

\begin{itemize}
\item 空のセルがないため、プレイヤーが可能な動きがない
\item どのプレイヤーの動きも、変異を起こすか、対戦相手に{\bf 勝っている}状態を与えてしまうかである。
\end{itemize}

こう考えると、ある状態が負けているのでなければ勝っているのだと言える。なぜなら、その状態において、変異も起こさず、対戦相手に勝っている状態を渡しもしない動きが1つ以上あるからである。
\end{paracol}
\vspace{\baselineskip}
\begin{paracol}{2}
To find the number of winning opening moves (if any) for Becca, we can check each move to see whether it is a winning move.

Of course, to do this, we have to investigate the resulting state recursively per the above definition.

However, since there are up to two moves per empty cell per state, the naive implementation that recursively counts the number of winning moves for each state may not be fast enough to handle even the $4\times 4$ grids in test set 1, so we should optimize it.
\switchcolumn
ベッカの勝利する最初の動きの個数を(もしあれば)見つけるために、私達はそれぞれの動きについて、勝利する動きかチェックできる。

もちろん、それを行うためには、上記の定義ごとに結果の状態を再帰的に調べなければならない。

だが、それぞれの状態について、空のセルごとに、可能な2つの動きがあるから、それぞれの状態の勝っている動きの数を再帰的に数えるための素朴な実装は、テストセット1の$4\times 4$のグリッドを扱うためにすら十分に速くはないかもしれない。よって私達はこれを最適化せねばならない。
\end{paracol}
\vspace{\baselineskip}
訳注。例えばごく単純化して考えて、$4\times 4$の16個のセルを2者が互いに選ぶとする。最初の選び方が16通り、次の相手の選び方が15通り、と続くが、16 = \seqsplit{%
20,922,789,888,000%
}であり、桁数で考えると$10^{13}$だ。解空間が広いので全探索が使えない。
\vspace{\baselineskip}
\begin{paracol}{2}
Notice that whether a state is winning or losing does not depend on who the player is or on any previous moves.

Since the same state may come up multiple times, we should consider memoizing our findings about each state to use in the future.

It may be daunting that the number of possible states is intractably large.

However, for any given case, there can be at most 16 initially empty cells, each of which can be either filled in by bacteria or not.

(After a colony has been placed and has spread, it no longer matters what type it was.)

So, we can put an upper bound of $2^{16}$ on the number of states per case.

In practice, there will be even fewer because not all states are reachable.
\switchcolumn
ある状態が勝っているか負けているかは、プレイヤーが誰か、および、それ以前の動きが何かに依存しないことに注意しよう。

同じ状態が複数回現れるかもしれないから、判定できたそれぞれの状態を将来に使うためにメモ化することを考えるべきだろう。

ありうる状態の数が扱いがたいほど大きいことが気力を削ぐかもしれない。

しかし、入力される任意のケースについて、最初の空のセルの個数は最大で16であり、そのそれぞれは、バクテリアで覆われているかいないかかだ。

(コロニーが置かれて広がってしまえば、コロニーの型は関係なくなる。)

よって、入力ケースあたりの状態の個数の上界として$2^{16}$を置ける。

実際には、すべての状態が到達可能なのではないから、状態の個数はさらに少ない。
\end{paracol}
\vspace{\baselineskip}
訳注。なお、$2^{16} = 65536$である。
\vspace{\baselineskip}
\begin{paracol}{2}
Moreover, we can save some time by not computing the exact number of winning moves for every state we examine.

We only care about this value for an initial state; for every other state, it suffices to determine whether it is winning or losing.

If we are investigating a non-initial state's moves and we find a winning move, we can declare the state to be winning, and stop.

This optimization alone may be enough to solve test set 1.
\switchcolumn
さらに私達は、調査するすべての状態についての勝っている動きの正確な個数を計算しないことで、いくらかの時間を節約できる。

初期状態についてしか私達はこの値に興味がない。他のすべての状態については、勝っているか負けているかさえ分かれば十分である。

もし私達が初期状態ではない状態の動きらを調査しており、勝っている動きを見つけたならば、その状態は勝っていると宣言して、終えることができる。

テストセット1を解くには、この最適化だけで十分かもしれない。
\end{paracol}
\subsection*{Test set 2 テストセット2}
\begin{paracol}{2}
When a player makes a legal move, the bacteria spread across the entire width or length of the row or column, up until the line of bacteria reaches the edge of the grid or a cell that is already infected.

Therefore, each move creates up to two subproblems that are independent in the sense that a move in one subproblem does not affect the state of the other.
\switchcolumn
あるプレイヤーが合法な動きを行ったとき、グリッドの端か、あるいはすでに感染しているセルに達しない限り、バクテリアは行か列の幅か高さ全体にまで広がる。

よって、それぞれの動きは、最大で2つのサブ問題を作成する。それらサブ問題は、片方のサブ問題での動きがもう片方の状態に影響しないという意味で、独立である。
\end{paracol}
\vspace{\baselineskip}
\begin{paracol}{2}
Each subproblem can be expressed as a rectangle contained within the full grid.

There are therefore at most O($\mathbf{R}^2\mathbf{C}^2$) subproblems.

How can we use the results of these subproblems to determine the overall winner of the game?
\switchcolumn
各サブ問題は、全体のグリッドに含まれるある長方形として表現することができる。

よって、最大でO($\mathbf{R}^2\mathbf{C}^2$)個のサブ問題が存在することになる。

それらのサブ問題の結果をどのようにして、ゲーム全体の勝者を決めるのに使うことができるだろう?
\end{paracol}
\vspace{\baselineskip}
訳注。R行C列の行列のなかにある長方形を考えるとき、(雑にオーダーを考えるとして、)上辺の選び方がR通り、下辺の選び方がR通り、また、左辺の選び方がC通り、右辺の選び方がC通りあるから、O($RRCC$)=O($R^2C^2$)である。
\vspace{\baselineskip}
\begin{paracol}{2}
The goal of the game is to force the opponent into a situation in which there is no move they can make that leads them down a path to victory.

The game is {\it impartial}: both players have access to the same set of moves.

It is therefore apt to draw a comparison between Bacterial Tactics and the ancient game Nim, an impartial game with similar types of decisions.

The mathematics of Nim are well-studied.

A discovery particularly useful to us is the Sprague-grundy Theorem, which says that any impartial game can be mapped to a state of Nim.

Every state in Nim corresponds to a non-negative {\it Grundy number}, or {\it nimber}, where any nonzero nimber indicates a winnable game.
\switchcolumn
ゲームの目的は、対戦相手をある状況、つまり、対戦相手の勝利につながる動きが選べない状況に追い込むことだ。

このゲームは{\bf 不偏}である。すなわち、選択しうる動きはどちらのプレイヤーも変わらない。

よって、Bacterial Tacticsと、似たタイプの決定をする不偏ゲームである古代のゲームNimとを比較したくなる。

Nimについての数理はよく研究されている。

私達にとって特に有用な発見は、スプレイグ・グランディの定理だ。任意の不偏ゲームは、Nimの状態に写像できるとされる。

Nimのすべての状態は、非負のグランディ数に対応する。任意のゼロでないグランディ数は、勝てるゲームを意味する。
\end{paracol}
\vspace{\baselineskip}
\begin{paracol}{2}
According to nimber addition, the nimber of a game state after we place a colony is equal to the {\it XOR} of the two subproblems.

The nimber of a game state before we place a colony is the {\it minimum excludant}, or {\it MEX}, of the set of possible nimbers after placing colonies.

We can therefore solve Bacterial Tactics recursively using the following pseudocode:
\switchcolumn
グランディ数の加算によると、私達がコロニーを置いた後のあるゲームの状態のグランディ数は、2つのサブ問題のXORに等しい。

私達がコロニーを置く前のゲームの状態のグランディ数は、コロニーらを置いたあとに可能なグランディ数らの補集合の最小値である。

よって私達は、次の疑似コードを使ってBacterial Tacticsを再帰的に解ける。
\end{paracol}

\begin{verbatim}
let solve(state) be a function:
  let s = Ø
  for each legal colony placement:
    add [solve(first subproblem) XOR solve(second subproblem)] to s
  return MEX(s)
\end{verbatim}

\begin{paracol}{2}
Given this general framework, we can now optimize our implementation.
\switchcolumn
この一般的なフレームワークがあるものとして、私達の実装を最適化することができる。
\end{paracol}
\vspace{\baselineskip}
\begin{paracol}{2}
First, as in test set 1, we can memoize the game states, which are now defined using rectangles of various sizes within the original grid.

Note that it is not possible to have bacteria from previous moves in a subproblem, because we always cut the rectangle along the row or column of cells infected by a colony placement.

We may also want to pre-compute the nimbers of all sizes of an empty rectangle (no radioactive cells), which is information that can be shared across all test cases.
\switchcolumn
第1に、テストセット1と同じく、ゲームの状態らをメモ化できる。各状態は今回は、オリジナルのグリッドのなかの様々な大きさの長方形で定義される。

なお、あるサブ問題では、以前の動きらから細菌を得ることは不可能だ。なぜなら、私達は常に、コロニーを置いて感染した行ないし列によって長方形を切断するからである。

私達はまた、(放射性物質のセルを含まない)すべての大きさの空の長方形のグランディ数を事前に算出しておきたいかもしれない。その情報は、テストケース全体で共有できる。
\end{paracol}
\vspace{\baselineskip}
\begin{paracol}{2}
Second, observe that if it is legal to place a V colony in a cell, then it is also legal to place a V colony in any cell in that column, and similarly for H colonies in a row, within the boundary of the current subproblems's rectangle.

We therfore need to check only the rows and columns for legal colony placements, not each individual cell.
\switchcolumn
第2に、現在のサブ問題の長方形の枠のなかで、あるセルにVコロニーを置くことが合法であるなら、同じ列のどのセルにVを置くこともまた合法であり、行についてはHコロニーについても同様に言えることに気づく。

よって私達は、コロニーの合法な設置のために行らや列らを調べるだけでよく、それぞれのセル自体を調べずにすむ。
\end{paracol}
\vspace{\baselineskip}
\begin{paracol}{2}
Third, we can construct a data structure that allows us to determine whether a colony placement is legal for any row or column in a given rectangle in O(1) time, allowing us to evaluate any game state in O({\bf R}+{\bf C}) operations.

For each row and column in the full grid, create an array.

Check the cells in the row or column in ascending order, appending the 1-indexed position of the most recently seen radioactive cell, or 0 if a radioactive cell has not been encountered yet.

For example, for the row .\#..\#, the array would be [0, 2, 2, 2, 5].

Suppose we have a rectangle that includes the third and fourth cells of that row.

The fourth entry of the array is a 2.

Since cell 2 is not in our rectangle (we have cells 3 and 4 only), we can conclude that it is safe to place an H colony in this row of our rectangle.

This data structure can be pre-computed for each test case in O({\bf RC}) time.
\switchcolumn
第3に、あるデータ構造を作成することで、与えられた長方形の任意の行や列について、コロニーの設置が合法かどうかをO(1)で決められる。これにより、任意のゲームの状態をO({\bf R}+{\bf C})回の操作で評価できる。

グリッド全体のそれぞれの行とそれぞれの列について配列を作る。

行または列のセルらを昇順にチェックする。最も最近見た放射性物質のセルの1-indexed位置を追加し、放射性物質のセルをまだ見てなければ0を追加する。

例えば、行.\#..\#については、配列は[0, 2, 2, 2, 5]となる。

その行の3番目および4番目のセルを含む長方形があったとしよう。

配列の4番目の要素は2だ。

2番目のセルは私達の長方形に入っていない(3番目と4番目だけ入っている)から、長方形のこの行にHコロニーを置くことは安全だと結論できる。

このデータ構造は、それぞれのテストケースについてO({\bf RC})時間で事前に算出できる。
\end{paracol}
\vspace{\baselineskip}
\begin{paracol}{2}
To summarize, there are O($\mathbf{R}^2\mathbf{C}^2$) subproblems, and each subproblem takes O({\bf R}+{\bf C}) operations.

If we let N be {\it max}({\bf R}, {\bf C}), this leads to O($N^5$) total time complexity, sufficient for test set 2.

Less efficient solutions might still pass, depending on their implementations.
\switchcolumn
まとめると、O($\mathbf{R}^2\mathbf{C}^2$)個のサブ問題が存在し、それぞれについてO({\bf R}+{\bf C})回の操作が必要だ。

Nを{\it max}({\bf R}, {\bf C})とすると、合計O($N^5$)時間の計算量となり、テストセット2に十分だ。

もっと非効率な解法であっても、実装によっては通過できるかもしれない。
\end{paracol}

\subsection*{implementation}
\verbatiminput{input/c1.py}

\subsection*{implementation}
\verbatiminput{input/73rd_Arios16_mod.py}


\vspace{\baselineskip}
\begin{paracol}{2}
\switchcolumn
\end{paracol}
\end{document}










