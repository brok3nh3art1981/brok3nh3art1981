%TEMPLATE-BEGIN 1
\documentclass[uplatex,dvipdfmx]{jsarticle} \usepackage{amsmath,amssymb,bm}
\usepackage{verbatim} \usepackage{svg} \usepackage{graphicx}
\usepackage[dvipdfmx]{hyperref} \usepackage{pxjahyper}
\usepackage{paracol} \usepackage{threeparttable}
\usepackage{seqsplit}
%TEMPLATE-END
\title{Learning Swift Language on Ubuntu:\\Methods\\Subscripts\\Inheritance\\Initialization} \author{} \date{}
\begin{document}
\maketitle


https://github.com/apple/swift

このGitリポジトリでSwift言語は開発されている。このリポジトリのdocsディレクトリに資料がある。

\section*{ASTとSILとIRの取得方法}
.swiftのコードは、AST (Abstract Syntax Tree)とSIL (Swift Intermediate Language)とLLVM IR (Low Level Virtual Machine Intermediate Representation)をこの順に経てアセンブリになる。それらを得る方法は次のものである。

\begin{itemize}
\item {\tt swiftc -dump-parse example.swift\ \ \ \ \ \# AST(構文木)を出力する}
\item {\tt swiftc -dump-ast example.swift\ \ \ \ \ \ \ \# 型チェック済みのASTを出力する}
\item {\tt swiftc -emit-silgen example.swift\ \ \ \ \# sil\_stage rawを出力する}
\item {\tt swiftc -emit-sil example.swift\ \ \ \ \ \ \ \# sil\_stage canonicalを出力する}
\item {\tt swiftc -emit-ir example.swift\ \ \ \ \ \ \ \ \# LLVM IRを出力する}
\item {\tt swiftc -emit-assembly example.swift\ \ \# アセンブリを出力する}
\end{itemize}




\section*{implementation}
\verbatiminput{input/011_methods.swift}

\section*{implementation}
\verbatiminput{input/012_subscripts.swift}

\section*{implementation}
\verbatiminput{input/013_inheritance.swift}

\section*{implementation}
\verbatiminput{input/014_initialization.swift}







\vspace{\baselineskip}
\begin{paracol}{2}
\switchcolumn
\end{paracol}
\end{document}










