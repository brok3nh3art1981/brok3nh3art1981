%TEMPLATE-BEGIN 1
\documentclass[uplatex,dvipdfmx]{jsarticle} \usepackage{amsmath,amssymb,bm}
\usepackage{verbatim} \usepackage{svg} \usepackage{graphicx}
\usepackage[dvipdfmx]{hyperref} \usepackage{pxjahyper}
\usepackage{paracol} \usepackage{threeparttable}
\usepackage{seqsplit}
%TEMPLATE-END
\title{Learning Swift Language on Ubuntu:\\Deinitialization\\Optional Chaining\\Error Handling\\Type Casting\\Nested Types\\Extensions\\Protocols} \author{} \date{}
\begin{document}
\maketitle
\section*{abstract 要旨}
Protocolという章の@objc属性に関する例示については、(「import Foundation」したとしても、)Ubuntuにおいて正常に動かなかった。

\section*{4種類のエラー}
Optionalの使い方について。エラーはすべてnilで流す感じで、開発時にはassert()で落とせばよいのではないか。

https://github.com/apple/swift/blob/master/docs/ErrorHandlingRationale.rst

というものがあるらしい。エラーには4種類ある。simple domain errorとrecoverble errorとuniversal errorとlogic failureである。simple domain errorにはOptional型でnil値を返し、recoverble errorには例外を投げ、universal errorはfatalError()で終了し、logic failureはprecondition()すればいい。なおエラーの原因が1つではない場合に、recoverble errorとして例外を投げることが適している。原因が1つならばOptionalが便利である。

\section*{Optionalについてのある考え方}
こういう考え方ができるのではないだろうか。つまり、生じうるエラーの種類が1種類になるまで、処理を細かく関数らに分解する。すると、1個の関数は1個のエラーを生じることになる。そのエラーをOptional型の返り値nilによって表現する。関数は、どのような入力に対して自らがnilを返すかを、コメントにおいて明示する。ユーザはそれを読んでその関数を利用する。

返り値がOptionalであることは、APIとユーザの間のコミュニケーションであり、nilが返りうることについてユーザの注意を喚起しているにすぎない。ユーザは、決してnilが返らないように引数を処理してから関数に渡す。このようにして、関数がnilを返さないことは、APIのコメントとユーザの思考によって確認され、機械的には確認されない。

ユーザは自らの責任において関数の返り値をすぐにbang (x!)する。nilをbangして実行時エラーになるなら、APIの説明か自身の考えたロジックに欠陥があったということになる。1つ1つのbangについて、ユーザはそれが正当である理由をすぐ近くにコメントで説明する。「bang because ...」といった記述が並ぶことになる。

エラーを生じるうる複数の関数を利用する関数で生じうるエラーが1種類だとは考えられない。よって関数のエラーの種類を1種類にするとは、普遍的なまでの原則ではない。単に、とても重要な指針と見なすのみである。

ネットワークアクセスなど、いかにも多様なエラーが生じそうな場合には、例外によって呼び出し元とコミュニケーションする。そうではなく、上記の趣旨がうまく適合しない場合には、precondition()によってプログラムをエラー終了してしまう。そのようにして、正常系でないコードを各部分に最小に押し込めることによって、コードの簡潔さを保ち、総合的な生産性に寄与する。特に巨大ないしmission criticalなプログラムについては別である。


\section*{implementation}
\verbatiminput{input/015_deinitialization.swift}

\section*{implementation}
\verbatiminput{input/016_optional_chaining.swift}

\section*{implementation}
\verbatiminput{input/017_error_handling.swift}

\section*{implementation}
\verbatiminput{input/018_type_casting.swift}

\section*{implementation}
\verbatiminput{input/019_nested_types.swift}

\section*{implementation}
\verbatiminput{input/020_extensions.swift}

\section*{implementation}
\verbatiminput{input/021_protocols.swift}






\vspace{\baselineskip}
\begin{paracol}{2}
\switchcolumn
\end{paracol}
\end{document}










