%TEMPLATE-BEGIN 1
\documentclass[uplatex,dvipdfmx]{jsarticle} \usepackage{amsmath,amssymb,bm}
\usepackage{verbatim} \usepackage{svg} \usepackage{graphicx}
\usepackage[dvipdfmx]{hyperref} \usepackage{pxjahyper}
\usepackage{paracol} \usepackage{threeparttable}
\usepackage{seqsplit}
%TEMPLATE-END
\title{elastic therapeutic tape キネシオテープ} \author{} \date{}
\begin{document}
\maketitle
\begin{paracol}{2}
{\bf Elastic therapeutic tape}, also called {\bf kinesiology tape}, {\bf Kinesio tape}, {\bf k-tape}, or {\bf KT}, is an elastic cotton strip with an acrylic adhesive that is used with the intent of treating pain and disability from athletic injuries and a variety of other physical disorders.

In individuals with chronic musculoskeletal pain, research suggests that elastic taping may help relieve pain, but not more than other treatment approaches, and there is no evidence that it can reduce disability in chronic pain cases.
\switchcolumn
伸縮性療法的テープとは、(キネシオロジーテープ、キネシオテープ、kテープ、KTとも呼ばれ、)アクリル性の接着剤を備えた伸縮性の綿の帯である。これは、運動競技による怪我や、様々なその他の肉体的な症状について、痛みや障害を治療する意図で用いられる。

筋骨格の慢性的な痛みを持つ人々について、伸縮性のテーピングが痛みを和らげる場合があることを研究は示唆しているが、その効果は他の治療法と同程度であり、慢性的な痛みがある場合には、障害を軽減することについて証拠は存在しない。
\end{paracol}
\vspace{\baselineskip}
\begin{paracol}{2}
The medical and scientific skeptical communities have evaluated the benefits of KT, and found no convincing scientific evidence that it provides any demonstrable benefit in excess of a placebo, declaring it a pseudoscientific treatment.
\switchcolumn
医学および科学的な懐疑的なコミュニティがKTの有効性を評価した。その結果、KTは、プラセボを越える有効性を実証しなかったため、妥当な科学的な証拠は見つからず、それを疑似科学による治療法であると宣言した。
\end{paracol}
\section*{History 歴史}
略略略略略略略略略略略略略略略略略略略略略略略略略略略略略略略略略略略略略略略略略略略略略略略略略略略略略略略略略略略略略略略略略略略略略略略略略略略略略略略略略略略略略略略略略略略略略略略略略略略略





\vspace{\baselineskip}
\begin{paracol}{2}
\switchcolumn
\end{paracol}
\end{document}










