%TEMPLATE-BEGIN 1
\documentclass[uplatex,dvipdfmx]{jsarticle} \usepackage{amsmath,amssymb,bm}
\usepackage{verbatim} \usepackage{svg} \usepackage{graphicx}
\usepackage[dvipdfmx]{hyperref} \usepackage{pxjahyper}
\usepackage{paracol} \usepackage{threeparttable}
\usepackage{seqsplit}
%TEMPLATE-END
\title{kinesiology キネシオロジー} \author{} \date{}
\begin{document}
\maketitle
\begin{paracol}{2}
{\bf Kinesiology} is the scientific study of human or non-human body movement.

Kinesiology addresses physiological, biomechanical, and psychological dynamic principles and mechanisms of movement.

Applications of kinesiology to human health (i.e., {\bf human kinesiology}) include biomechanics and orthopedics; strength and conditioning; sport psychology; methods of rehabilitation, such as physical and occupational therapy; and sport and exercise.

Studies of human and animal motion include measures from motion tracking systems, electrophysiology of muscle and brain activity, various methods for monitoring physiological function, and other behavioral and cognitive research techniques.
\switchcolumn
{\bf キネシオロジー}とは、人間または人間以外の生物の身体の運動についての科学的な研究である。

キネシオロジーが議論する範囲には、生理学的、生体力学的、心理学的な運動原理および運動機構が含まれる。

({\bf ヒューマン・キネシオロジー}などの、)キネシオロジーの人間の健康への適用に含まれるのは、生体力学と整形外科学、体力とコンディショニング、スポーツ心理学、肉体的または職業的なセラピーなどのリハビリテーションの方法、スポーツとエクササイズなどである。

人間や動物の運動についての研究に含まれるのは、モーショントラッキングシステムを用いた測定尺度、筋肉や脳の活動についての電気生理学、生理学的な機能を観測するために用いられる様々な方法、またその他に、行動と認知について研究する技術などがある。
\end{paracol}
\vspace{\baselineskip}
\begin{paracol}{2}
The word comes from the Greek κ\'ινησις k\'in\=esis, ``movement" (itself from κινε\~ιν kine\^in, ``to move"), and -λογ\'ια -logia, ``study". 
\switchcolumn
キネシオロジーという言葉は、ギリシャ語の(動くこと、κινε\~ινに由来する)κ\'ινησις キネシス(k\'in\=esis、動き)、λογ\'ια-ロギア(logia、研究・学問)から来ている。
\end{paracol}
\section*{Basics 基本}
\begin{paracol}{2}
Kinesiology is the study of human and nonhuman animal-body movements, performance, and function by applying the sciences of biomechanics, anatomy, physiology, psychology, and neuroscience.

Applications of kinesiology in human-health include physical education teacher, rehabilitation, health and safety, health promotion, workplaces, sport and exercise industries.

A bachelor's degree in kinesiology can provide strong preparation for graduate study in biomedical research, as well as in professional programs, such as medicine.
\switchcolumn
キネシオロジーとは、人間あるいは人間でない動物の身体の、動きやパフォーマンスや機能について、生体力学、解剖学、生理学、心理学、神経科学などの科学を適用することによって研究を行う学問領域である。

人間の健康に関するキネシオロジーの応用分野としては、教育における運動の教員、リハビリテーション、健康と安全、健康の増進、職場、スポーツ、エクササイズの業界などがある。

キネシオロジーで学士の学位を得ることは、生体医療についての大学院での研究の強力な準備にもなるし、医学などのプロフェッショナルなプログラムの準備にもなる。
\end{paracol}
\vspace{\baselineskip}
\begin{paracol}{2}
Whereas the term ``kinesiologist'' is neither a licensed nor professional designation in the United States nor most countries (with the exception of Canada), individuals with training in this area can teach physical education, provide consulting services, conduct research and develop policies related to rehabilitation, human motor performance, ergonomics, and occupational health and safety.

In North America, kinesiologists may study to earn a Bachelor of Science, Master of Science, or Doctorate of Philosophy degree in Kinesiology or a Bachelor of Kinesiology degree, while in Australia or New Zealand, they are often conferred an Applied Science (Human Movement) degree (or higher).

Many doctoral level faculty in North American kinesiology programs received their doctoral training in related disciplines, such as neuroscience, mechanical engineering, psychology, and physiology.
\switchcolumn
「キネシオロジスト」という語は、米国においても(カナダを除く)ほとんどの国々においても、ライセンスされたものでもプロフェッショナルな称号でもないが、この領域で訓練された人々は、肉体的な教育を教えることができ、コンサルティングのサービスを行うことができ、研究を行えるし、リハビリテーション、人間の動力のパフォーマンス、人間工学、職業的な健康と暗線について、ポリシーを開発できる。

北米では、キネシオロジストは、理学学士を得るため、理学修士を得るため、哲学博士(Ph.D.)を得るために研究されたり、または、キネシオロジー学士を得るために研究される。一方、オーストラリアとニュージーランドではそういった研究に対して、人間の動作についての準学士(またはそれ以上のもの)が与えられる。

北米におけるキネシオロジーのプログラムの多くの博士レベルの過程が、神経科学、機械工学、心理学、生理学などといった、関連する学科における博士的な訓練を通して行われる。
\end{paracol}
\vspace{\baselineskip}
\begin{paracol}{2}
The world's first kinesiology department was launched in 1967 at the University of Waterloo, Canada.
\switchcolumn
世界で最初のキネシオロジーの部署は、カナダのウォータールー大学に1967年に設置されたものである。
\end{paracol}
\section*{Principles 原理}
\subsection*{Adaptation through exercise エクササイズによる適応}
\begin{paracol}{2}
Adaptation through exercise is a key principle of kinesiology that relates to improved fitness in atheletes as well as health and wellness in clinical populations.

Exercise is a simple and established intervention for many movement disorders and musculoskeletal conditions due to the neuroplasticity of the brain and the adaptability of the musculoskeletal system.

Therapeutic exercise has been shown to improve neuromotor control and motor capabilities in both normal and pathological populations.
\switchcolumn
エクササイズを通した適応は、キネシオロジーで鍵になる原則であり、運動選手のフィットネスの改善、および、臨床の人々のウェルネスの改善に関係する。

エクササイズとは、多くの運動障害に対して、あるいは筋骨格疾患に対して実施するための、シンプルで確立された介入である。それらの障害や疾患は、脳の神経可塑性や筋骨格系の適応性を原因として発生する。

健康な人々についても病理的な人々についても、療法的なエクササイズは、神経運動の制御および運動の能力について、改善させることを示してきた。
\end{paracol}
\vspace{\baselineskip}
\begin{paracol}{2}
There are many different types of exercise interventions that can be applied in kinesiology to athletic, normal, and clinical populations.

Aerobic exercise interventions help to improve cardiovascular endurance.

Anaerobic strength training programs can increase muscular strength, power, and lean body mass.

Decreased risk of falls and increased neuromuscular control can be attributed to balance intervention programs.

Flexibility programs can increase functional range of motion and reduce the risk of injury.
\switchcolumn
様々な種類のエクササイズの介入が存在し、運動競技、一般の人々、臨床の治療のもとにある人々に適用できる。

有酸素運動の介入は、心臓の血管の耐性を向上させる支援となる。

無酸素運動による体力の強化を行ったならば、筋肉の力の強さを増大する可能性があり、体重を増加する傾向がある。

バランスについての介入プログラムの性質は、落ち込みが起こるリスクを減らし、神経筋のコントロールを増加させることにある。

柔軟性プログラムは、運動の機能的な範囲を増加し、怪我の危険を減らすものである。
\end{paracol}
\vspace{\baselineskip}
\begin{paracol}{2}
As a whole, exercise programs can reduce symptoms of depression and risk of cardiovascular and metabolic diseases.

Additionally, they can help to improve quality of life, sleeping habits, immune system function, and body composition.
\switchcolumn
全体としては、エクササイズのプログラムによって、抑鬱の症状を軽減でき、心血管疾患やメタボリック疾患の危険を減らすことができる。

そしてさらに、生活の質の改善、睡眠の習慣の改善、免疫機能の改善、姿勢の改善に役立つ可能性がある。
\end{paracol}
\vspace{\baselineskip}
\begin{paracol}{2}
The study of the physiological responses to physical exercise and their therapeutic applications is known as exercise physiology, which is an important area of research within kinesiology.
\switchcolumn
肉体的なエクササイズとその療法的な応用に対する生理学的な反応の研究は、エクササイズ生理学として知られており、キネシオロジーの重要な研究領域である。
\end{paracol}
略略略略略略略略略略略略略略略略略略略略略略略略略略略略略略略略略略略略略略略略略略略略略略略略略略略略略略略略略略略略略略略略略略略略略略略略略略略略略略略略略略略略略略略略略


\vspace{\baselineskip}
\begin{paracol}{2}
\switchcolumn
\end{paracol}
\end{document}










