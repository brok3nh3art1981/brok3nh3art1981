%TEMPLATE-BEGIN 1
\documentclass[uplatex,dvipdfmx]{jsarticle} \usepackage{amsmath,amssymb,bm}
\usepackage{verbatim} \usepackage{svg} \usepackage{graphicx}
\usepackage[dvipdfmx]{hyperref} \usepackage{pxjahyper}
\usepackage{paracol} \usepackage{threeparttable}
\usepackage{seqsplit}
%TEMPLATE-END
\title{The Perceptron: A Probablistic Model for \\Information Storage and Organization in the Brain\footnote{
The development of this theory has been carried out at the Cornell Aeronautical Laboratory, Inc., under the sponsorship of the Office of Naval Research, Contract Nonr-2381 (00). This article is primarily an adaptation of material reported in Ref. 15, which constitutes the first full report on the program.
この理論の発達は海軍研究所の契約Nonr-2381(00)の資金援助のもと、コーネル大学航空研究所で行われた。この論稿は第一に、このプログラムについての同じ著者による最初の完全な報告である参照15の翻案である。
}
\\パーセプトロン: 脳における情報の保存と\\組織化についての確率的モデル} \author{Frank Rosenblatt (1928-1971)} \date{}
\begin{document}
\maketitle

\begin{paracol}{2}
\vspace{\baselineskip}
If we are eventually to understand the capability of higher organisms for perceptual recognition, generalization, recall, and thinking, we must first have answers to three fundamental questions:
\begin{enumerate}
\item How is information about the physical world sensed, or detected, by the biological system?
\item In what form is information stored, or remembered?
\item How does information contained in storage, or in memory, influence recognition and behavior?
\end{enumerate}
\switchcolumn
\vspace{\baselineskip}
知覚による認識、一般化、思い出し、思考のための、高級な有機的組織の能力をもし私達がやがて理解することになるのであれば、私達は先立ってまず、次の基本的な疑問に答えねばならない:
\begin{enumerate}
\item 物質的な世界についての情報はいかにして、生物学的な系に感覚されたり知覚されたりするのか?
\item どのような形で、情報の保存や記憶は行われているのか?
\item ストレージや記憶に保持されている情報は、どのようにして、認識や行動に影響しているのか?
 \end{enumerate}
\end{paracol}
\begin{paracol}{2}
The first of these questions is in the province of sensory physiology, and is the only one for which appreciable understandings has been achived.

This article will be concerned primarily with the second and third questions, which are still subject to a vast amount of speculation, and where the few relevant facts currently supplied by neurophysiology have not yet been integrated into an acceptable theory.
\switchcolumn
これらの疑問の1つ目のものは、感覚についての生理学の職分だ。そしてその疑問についてだけは、明確な理解が達成されてきた。

この論考では、2つ目と3つ目の疑問を中心に考える。それらはまだ、大いに憶測のなかにある議題である。そしてそこにおいては、関連する事実が神経生理学によってわずかにもたらされているものの、受容できる統合された理論には今はまだ存在していない。
\end{paracol}
\vspace{\baselineskip}
\begin{paracol}{2}
With regard to the second question, two alternative positions have been maintained.

The first suggests that storage of sensory information is in the form of coded representations or images, with some sort of one-to-one mapping between the sensory stimulus and the stored pattern.

According to this hypothesis, if one understood the code or ``wiring diagram'' of the nervous system, one should, in principle, be able to discover exactly what an organism remembers by reconstructing the original sensory patterns from the ``memory traces'' which they have left, much as we might develop a photographic negative, or translate the pattern of electrical charges in the ``memory'' of a digital computer.

This hypothesis is appealing in its simplicity and ready intelligibility, and a large family of theoretical brain models has been developed around the idea of a coded, representational memory.

The alternative approach, which stems from the tradition of British empiricism, hazards the guess that the images of stimuli may never really be recorded at all, and that the central nervous system simply acts as an intricate switching network, where retention takes the form of new connections, or pathways, between centers of activity.

In many of the more recent developments of this position (Hebb's ``cell assembly,'' and Hull's ``cortical anticipatory goal response,'' for example) the ``responses'' which are associated to stimuli may be entirely contained within the CNS itself.

In this case the response represents an ``idea'' rather than an action.

The important feature of this approach is that there is never any simple mapping of the stimulus into memory, according to some code which would permit its later reconstruction.

Whatever information is retained must somehow be stored as a {\it preference for a particular response;} i.e., the information is contained in {\it connections} or {\it associations} rather than topographic representations.

(The term {\it response}, for the remainder of this presentation, should be understood to mean any distinguishable state of the organism, which may or may not involve externally detectable muscular activity.

The acttivation of some nucleus of cells in the central nervous system, for example, can be constitute a response, according to this definition.)
\switchcolumn
2つ目の疑問については、2つの異なる立場が存在してきた。

第1の立場は、知覚された情報について、符号化された表現やイメージだとする。そこでは、何らかの種類の、知覚刺激から保存されるパターンへの1対1のマッピングが存在する。

この仮説によると、
\end{paracol}





\vspace{\baselineskip}
\begin{paracol}{2}
\switchcolumn
\end{paracol}
\end{document}










