%TEMPLATE-BEGIN 1
\documentclass[uplatex,dvipdfmx]{jsarticle} \usepackage{amsmath,amssymb,bm}
\usepackage{verbatim} \usepackage{svg} \usepackage{graphicx}
\usepackage[dvipdfmx]{hyperref} \usepackage{pxjahyper}
\usepackage{paracol} \usepackage{threeparttable}
\usepackage{seqsplit}
%TEMPLATE-END
\title{Irrlicht Engineについて} \author{} \date{}
\begin{document}
\maketitle



\section*{Irrlicht Engine}
と書いて、「イアリハト・エンジン」と読む3Dエンジンがあるらしい。今では人気がなさそうだが、かつて日本語の情報がいくらかあったようだ。aptコマンドでインストールできる点が気になったので、少し触れてみたい。開発者は、Nikolaus Gebhardt。


\section*{installation}
\begin{verbatim}
sudo apt install libirrlicht-dev
\end{verbatim}

\section*{依存性}
他にも下記をインストールした。

\begin{verbatim}

build-essential
codeblocks
freeglut3-dev
libboost-dev
libfreetype6-dev
libgl1-mesa-dev
libglew1.5-dev
libglu1-mesa-dev
libopenal-dev
libtheora-dev
libxcursor-dev
libxext-dev
libxxf86vm-dev
mesa-common-dev
xserver-xorg-dev
x11proto-xf86vidmode-dev
\end{verbatim}


\section*{simpgさんのhello, world}
http://d.hatena.ne.jp/simpg/20131012 にあった次のコードは、g++ example.cpp -lIrrlicht としてコンパイルでき動作した。-lGLもつけたほうがよい場合がありそうである。

\verbatiminput{input/001/001.cpp}

\section*{終了方法}

Alt+F4で終了できる。


\vspace{\baselineskip}
\begin{paracol}{2}
\switchcolumn
\end{paracol}
\end{document}
